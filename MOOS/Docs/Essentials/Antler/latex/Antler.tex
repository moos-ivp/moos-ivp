

\documentclass[a4paper,10pt]{article}
\usepackage{listings,color,epsfig,amsmath,url}
\definecolor{codecolor}{rgb}{0.99,0.97,0.94} % color values Red, Green, Blue
\definecolor{commentcolor}{rgb}{0.1,0.5,0.1} % color values Red, Green, Blue
\definecolor{stringcolor}{rgb}{0.3,0.1,0.1} % color values Red, Green, Blue
\newcommand{\Code}[1]{\texttt{#1} }
\newcommand{\code}[1]{\Code{#1} }
\newcommand{\DB}   {\code{{MOOSDB}}}
\newcommand{\MA}   {\code{{MOOSApp}}}
\newcommand{\Ignore}[1]   {}



% Title Page
\title{Launching MOOS Processes and Mission scripts with \code{pAntler}}
\author{Paul Newman}


\begin{document}
\maketitle

\begin{center}
\epsfig{file=images/moose6.eps,width = 0.2\linewidth} 
\end{center}
\begin{abstract}
This document tell you how to use the application \code{pAntler} to launch multiple MOOS processes. This is usefultool for starting up a whole bunck of processes all of which sahre a single configuration file.
\end{abstract}


\section{Introduction}
The process \code{pAntler} is used to launch/create a MOOS
community. It is very simple and very useful. It reads from its
configuration block a list of process names that will constitute
the MOOS community. Each process to be launched is specfied with a
line with the syntax
\begin{center}
{\code{Run}} = {\it{procname}} [ @ {\code{NewConsole}} =
{\it{true/false}}] [$\sim$ \code{MOOSName}]
\end{center}
The optional console parameter specifies whether the named process
should be launched in a new  window (an xterm in Unix or
cmd-prompt in NT derived platforms). Each process launched is
passed the mission file name  as a command line argument. When the
processes have been launched \code{pAntler} waits for all of the
community to exit and then quits itself.

\subsection{Running Multiple Instances of a Process}

The optional \code{MOOSName} parameter allows MOOSProcesses to
connect to the \DB under a specified name. For example a vehicle
may have two GPS instruments onboard. Now by default \code{iGPS}
may register it existence with the \DB under the name ``iGPS''. By
using this syntax multiple instances of the executable \code{iGPS}
can be run but each connects to a the \DB using a different name.

\begin{center}
\code{Run = iGPS @ NewConsole = true $\sim$ iGPSA}\\
\code{Run = iGPS @ NewConsole = true $\sim$ iGPSB}\\
\end{center}

would launch two instances of \code{iGPS} registering under
``iGPSA'' and ``iGPSB'' respectively. Note there would need to be
\emph{two} GPS configuration blocks in the mission file -- one for
each and the processnames would be ``iGPSA'' and ``iGPSB''


\section{Launching Missions}
\code{pAntler} provides a simple and compact way to start a MOOS
session. For example if the desired mission file is
{\it{Mission.moos}} then executing
\begin{center}
\code{pAntler Mission.moos}
\end{center}
will launch the required processes for the  mission. Of course a
sensibly designed mission will not actually start to do anything
until a human (via \code{iRemote}) has confirmed a good working
status of the processes involved (eg \code{pNav}) and actively
hands control over to the Helm.

\subsection{I/O Redirection - Deployment} As already mentioned,
frequently \code{iRemote}, displayed on a remote machine, will be
the only interface a mission pilot has to the MOOS community.  We
must ask the question -  ``where does all the IO from other
processes go to prevent I/O blocking?''. One answer to this is I/O
redirection and backgrounding MOOS processes - a simple task in
unix derived systems \footnote{some OS are good for development
others for running...}/

Running \code{pAntler} in the following fashion followed by a
manual start up of \code{iRemote} is the recommended way of
running MOOS in the field using a serial port login.

\begin{enumerate}
\item \code{pAntler} {\it{mission.moos}} $>$ ptyZ0 $>$ /dev/null \&
\item \code{iRemote} {\it{mission.moos}}
\end{enumerate}

This redirection of \code{iRemote} is encapsulated in the
\code{moosbg} script included with the MOOS installations. In the
case of an AUV the interface can only be reached  through in-air
wireless communications, which will clearly disappear when the
vehicle submerges but will gracefully re-connect when surfacing(
not so easy to do with a PPP or similar link).
\end{document}